\documentclass[10pt, a4paper]{article}
\usepackage{latexsym}
\usepackage{amssymb,amsmath}
\usepackage[pdftex]{graphicx}
\usepackage{float}
\newcommand{\dbar}[1]{\Bar{\Bar{#1}}}

\usepackage{hyperref}
\usepackage{caption}
\captionsetup[table]{position=bottom} 

\topmargin = 0.1in \textwidth=5.7in \textheight=8.6in

\oddsidemargin = 0.1in \evensidemargin = 0.1in

% headers
\usepackage{fancyhdr}
\pagestyle{fancy}
\chead{} 
\rhead{\thepage} 
% footer
\lfoot{\small\scshape } 
\cfoot{} 
%%%% insert your name here %%%%
\rfoot{\footnotesize Michael Burton} 
\renewcommand{\headrulewidth}{.3pt} 
\renewcommand{\footrulewidth}{.3pt}
\setlength\voffset{-0.25in}
\setlength\textheight{648pt}

\begin{document}

\title{Scaling the JHO for Larger Payloads}
\author{Michael Burton}
\maketitle

This study shows a scaled version of the Jungle Hawk Owl (JHO) that would accomodate a 100 lbs payload drawing 650 W.  
The same GPkit optimization models and basic assumptions used to size and optimize the JHO were used to do this study. 

An initial sizing analysis shows that a scaled verison of the JHO will be roughly twice as big and will have a max takeoff weight of 3 times the current aircraft max takeoff weight.  Figure~\ref{f:drawing} shows a scaled comparision of the current JHO and the resized verison. Table~\ref{t:keyvals} lists the estimated values of the scaled version.  For reference, the current aircraft's parameters are also listed.  Because a ``rubber'' engine was assumed in this study, availability of an engine and alternator to the specification listed in Table~\ref{t:keyvals} needs to be confirmed. 

\begin{figure}[H]
    \begin{center}
        \includegraphics[width=1.0\textwidth]{drawing.jpg}
        \caption{\textbf{Comparison of JHO and scaled version copable of supporting larger payloads.}}
        \label{f:drawing}
    \end{center}
\end{figure}

\begin{table}[H]
    \centering
    \begin{tabular}{lcc}
        Variable          & Scaled JHO      & JHO          \\ \hline \hline
        Payload Weight    & 100 [lbs]       & 10 [lbs]     \\
        Payload Power     & 650 [W]         & 100 [W]      \\
        MTOW              & 517 {[}lbs{]}   & 150 [lbs]    \\ 
        Empty weight      & 262 {[}lbs{]}   & 53.6 [lbs]   \\ 
        Span              & 44.5 {[}ft{]}   & 24 [ft]      \\ 
        Aircraft length   & 20 {[}ft{]}     & 12.9 [ft]    \\ 
        Aspect ratio      & 26.6            & 25.6         \\ 
        Root chord        & 2.22 {[}ft{]}   & 1.25 [ft]    \\ 
        Engine weight     & 18 {[}lbs{]}    & 7 [lbs]      \\ 
        Max engine power  & 16 {[}hp{]}     & 5 [hp]       \\ 
        Max speed         & 147 {[}kts{]}   & 106 [kts]    \\ 
        Loiter speed      & 49 {[}kts{]}    & 49 [kts]     \\ 
    \end{tabular}
    \caption{Key Sizing Parameters}
    \label{t:keyvals}
\end{table}

Because the aicraft size is sensitive to payload weight and endurance, understanding how those parameters trade with aircraft weight is helpful to understanding the design space.  
Figure~\ref{f:endurancetrade} shows countour lines of endurance requirements plotted against payload weight vs max take off weight.  
Each point on Figure~\ref{f:endurancetrade} is a different aircraft, optimized for that particular payload weight and endurance requirement. 

\begin{figure}[H]
    \begin{center}
        \includegraphics[width=0.7\textwidth]{endurancetrade.pdf}
        \caption{\textbf{Trade of max takeoff weight vs payload weight.  Contours are endurance requirements.}}
        \label{f:endurancetrade}
    \end{center}
\end{figure}

\subsubsection*{Basic Assumptions}

\begin{enumerate}

    \item Mission profile:
        \begin{enumerate}
            \item Climb to 15,000 ft
            \item Cruise 200 nmi to station
            \item Loiter for 5 days at 15,000 ft
            \item Cruise 200 nmi to base
            \item Descend 
            \end{enumerate}
    \item 95th percentile world wide winds at 15,000 ft: 49 kts
    \item ``Rubber engine''; engine weight scales with required power

\end{enumerate}

For more information about Geometric Programming optimization or GPkit see: \url{hoburg.mit.edu}

\end{document}
